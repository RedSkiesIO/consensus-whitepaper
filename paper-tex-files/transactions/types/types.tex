On the Catalyst ledger, a transaction is a message or object used to transfer KAT tokens or data from and to a set of digital accounts. Such transaction can include different types of transfer depending on the nature of the accounts embedded in said transaction. As mentioned in section~\ref{Sec:AoC}, Catalyst supports the transfer of confidential and non-confidential assets. Catalyst also supports the transfer of assets and data linked to smart contracts and data storage. These different type of transfers are defined by specific transaction components that allow any node on the network to differentiate between the nature of exchanges embedded in different transactions. In this section we give an overview of the transaction structure and the different components considered for each type of token and data exchange.  